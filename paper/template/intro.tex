% In this section, you should introduce the reader to the problem you
% are attempting to solve. For example, for the first project: describe
% the dataset, and the prediction problem that you are
% investigating. You should also cite and briefly describe other related
% papers that have tackled this problem (or similar ones) in the past
% --- things that came up during the course of your research. In the
% AAAI style, citations look like \cite{aima} (see
% the comments in the source file \texttt{intro.tex} to see how this
% citation was produced). Conclude by summarizing how the
% remainder of the paper is organized. \\

% % Citations: As you can see above, you create a citation by using the
% % \cite{} command. Inside the braces, you provide a "key" that is
% % uniue to the paper/book/resource you are citing. How do you
% % associate a key with a specific paper? You do so in a separate bib
% % file --- for this document, the bib file is called
% % project1.bib. Open that file to continue reading...

% % Note that merely hitting the "return" key will not start a new line
% % in LaTeX. To break a line, you need to end it with \\. To begin a 
% % new paragraph, end a line with \\, leave a blank
% % line, and then start the next line (like in this example).
% Overall, the aim in this section is context-setting: what is the
% big-picture surrounding the problem you are tackling here?

% The \section{} command formats and sets the title of this
% section. We'll deal with labels later.
\section{Introduction}
\label{sec:intro}

Spacecraft health monitoring relies critically on the continuous analysis of telemetry data streams to detect anomalous behaviour that may indicate system malfunctions or impending failures. Traditional anomaly detection methods often depend on manually configured thresholds and rules, which struggle to capture complex, multivariate temporal patterns in modern spacecraft systems. This paper addresses the challenge of automated anomaly detection in spacecraft telemetry using unsupervised machine learning techniques applied to real data from the Soil Moisture Active Passive (SMAP) satellite.

Our dataset comprises multivariate telemetry streams from nine distinct channels representing various spacecraft subsystems including power, radiation, and temperature monitoring. Each channel contains time series data with multiple sensor dimensions, presenting both point anomalies (sudden deviations) and contextual anomalies (patterns that are unusual within their temporal context). The data includes 55 manually labelled anomaly sequences across 69 unique telemetry channels, with 62\% classified as point anomalies and 38\% as contextual anomalies.

Previous work in spacecraft anomaly detection has explored various approaches, including statistical methods, deep learning architectures, and hybrid techniques. However, many supervised approaches require extensive labelled anomaly data, which is often scarce in operational spacecraft systems. Our work adopts an unsupervised learning paradigm that can identify anomalies without relying on labelled training examples.

We present an ensemble approach combining One-Class Support Vector Machine, Isolation Forest, and Local Outlier Factor algorithms. Our preprocessing pipeline employs overlapping temporal windows to capture contextual information and MiniRocket feature transformation to extract discriminative patterns from the multivariate time series. By systematically optimising detection thresholds to prioritise recall whilst maintaining reasonable precision, we demonstrate that unsupervised ensemble methods can achieve robust anomaly detection across diverse telemetry channels.

The remainder of this paper is organised as follows: Section 2 describes our dataset and preprocessing methodology; Section 3 details the feature engineering and classification techniques employed; Section 4 presents experimental results and analysis; and Section 5 concludes with discussion of limitations and future work.